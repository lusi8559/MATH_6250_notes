\documentclass{article}
\title{Differential Geometry Notes}
\author{Lucas Simon, }
\date{\copyright\ 2015 Markus Pflaum, All Rights Reserved} 

\usepackage{amsmath}
\usepackage{amsfonts}
\usepackage{amssymb}
\usepackage{amsthm}
\usepackage{hyperref}
\usepackage{tikz}
\usepackage{tikz-cd}
\usepackage{mathrsfs}
\usepackage[all]{xy}

\newcommand{\cat}[1]{\textbf{#1}}
\newcommand\Lie{\mathcal{L}}
%\newcommand{\det}[1]{\text{det}(#1 )}

\begin{document}

\maketitle

\section{Notice}
These notes contain errors. Please put an issue on github or just fork the repository and make the changes yourself.

\section{March 6, 2015}

If we have $f:M \to N$ and $q \in N$. The claim was $f^{-1}(q) \subset M$ is a submanifold. We have to find charts. Let $p \in f^{-1}(q)$. By the rank theorem on charts, there are charts $(U, x)$ around $p$ and $(V,y)$ around $q$ such that $x(p) = 0$ and $y(q) = 0$, and $y \circ f \circ x^{-1}(v)=(v_1, \ldots, v_n)$. for $v$ in a neighborhood of origin in $\mathbb{R}^m$. For $v$ in a neighborhood of origin in $\mathbb{R}^m$ the chart we are looking for, find $f^{-1}(q)$ is $(x_{n+1}, \ldots, x_m)$. Then $f^{-1}(q) \cap U = (x_{n+1}, \ldots, x_m)^{-1}(\mathbb{R}^{m-n})$

\textbf{Theorem (Ehresmann):} A proper surjective submersion is a fiber bundle with fiber

\textbf{Orientation:} Let $M$ be a smooth connected $m$-manifold. and let $\Lambda^kT^*M \sim O_N := \{ \omega_p \in \Lambda^kT^*M | p \in M \text{ and } \omega_p \neq 0\} \subset \Lambda^m T^*M$. These $\omega$'s are nonzero in some fiber. Then $\text{dim}\Lambda^m T^*_pM = 1$, this is the space of determinants. This is a subspace. A manifold $M$ is called \textbf{orientable} if $\Lambda^k T^*M \sim 0_M$ has exactly two connected components. An \textbf{orientation} of an orientable smooth manifold $M$ is a choice of a component of $\Lambda^kT^*M \sim 0_M$.

The zero section of a vector bundle $p:V \to M$ is a $M \to v, p \mapsto 0_p$ where $0_M$ is the zero section of $\Lambda^kT^*M$ or better its image. 

The tangent bundle recap: take $TM$ of some smooth manifold $M$, and let $(U,x)$ and $(V,y)$ be smooth charts such that $U \cap V \neq \varnothing$. Then $x \circ x^{-1}|_{U \cap V}:x(U \cap V) \to y(U \cap V)$ are smooth transition maps of $M$. Then the induced map on trivializations $x(U \cap V)\times \mathbb{R}^m \to y(U \cap V)\times \mathbb{R}^m$ where $(p,v) \mapsto (y \circ x^{-1}(p), D(y \circ x^{-1})(p)v)$

\textbf{Counterexample:} Mobius band.

\textbf{Example:} Chiral molecules have a defined orientation. Similar, but not the same.

\textbf{Theorem:} Let $M$ be a connected smooth $m$-manifold. Then the following are equivalent:

(1) $M$ is orientable

(2) There is an atlas $\mathcal{A}$ of $M$ such that the det($D(x \circ y^{-1})(y(p))) > 0$   for each $(U,x), (V,y) \in \mathcal{A}$ and $p \in U \cap V$

(3) There is a nowhere vanishing $m$-form $\omega \in \Omega^m(M)$.  

Proof: $(1) \Rightarrow (2)$ Let $\Lambda$ be an orientation have have $\Lambda \cap \Lambda^mT^*_pM \sim 0_p$ is a component of $\Lambda^mT^*_pM \sim 0_p$ Define $\mathcal{A}$ be the set of all charts $(U,x)$ of $M$ such that $dx_1 \wedge \cdots \wedge dx_m(p) \in \Lambda$ for all $p \in U$. Assume also that each $U$ is connected. We need to compute the transition functions to check ....WHAT?..... Let $(V,y)$ be a second chart from $\mathcal{A}$ such that $p \in U \cap V$ then 
\[
dx_1 \wedge \cdots \wedge dx_m |_p = \text{det}(\frac{\partial x_k}{\partial y_j})(p)dy_1 \wedge \cdots \wedge dy_m|p
\] 
Since $\text{det}(\frac{\partial x_k}{\partial y_j})(p) = \text{det}(D(x \circ y^{-1})(y(p))) > 0$. hence the transition functions are positive.

\section{March 9, 2015}

\textbf{Theorem} Let $M$ be a connected manifold. The following are equivalent

(1) $M$ is orientable

(2) There exists an atlas $\mathscr{A}$ of $M$ such that the detereminant of $D(x \circ y^{-1})(y(p)) > 0$ for all $(U,x), (V,y) \in \mathscr{A}$ and $p \in U \cap V$.

(3) There is a nowhere vanishing $\omega \in \Omega^m(M)$ with $m = \text{dim}(M)$.
\begin{proof}
$(2) \Rightarrow (3)$.  Under the hypotheses of $(2)$. Chose a smooth partition of unity $(\phi_i)_{i \in \mathbb{N}}$ of $M$ subordinate to $\mathscr{A}$; that is, for each $i \in \mathbb{N}$ there is $(U_i, x^{(i)}) \in \mathscr{A}$ such that $\text{supp}(\phi_i) \subset U$ is relatively compact; that is, it's closure is compact and contained in $U$. ($\sum \phi_i = 1$, (supp($\phi_i$)) is locally finite). Put $\omega := \sum_{i \in \mathbb{N}} \phi_i \cdot dx_1^{(i)} \wedge \cdots \wedge dx_m^{(i)}$. If $p \in U^{(i)} \cap U^{(j)}$, then $dx_1^{(i)} \wedge \cdots \wedge dx_m^{(i)} = \lambda_{ij}dx_1^{(j)}\wedge \cdots \wedge dx_m^{(j)}$ where $\lambda_{ij} = \text{det}(D(x^{(i)} \circ x^{(j)-1}) (x^{(j)}(p)) > 0$. Then $\omega(p) = (\sum_{i \in \mathbb{N}} \phi_j(p) \lambda_{ji}(p))dx_1^{(j)}\wedge \cdots \wedge dx_m^{(m)}(p)$. Since each term is greater than $0$.

$(3) \Rightarrow (1)$. Under the hypotheses of (3), there is a nowhere vanishing $\omega \in \Omega^m(M)$ such that $\omega(p)$ is nonzero for all $p \in M$. Then $\Lambda^mT^*M \sim 0_m$ is the union of $\Lambda^+ := \{ \rho \in \lambda^m T^*M : \rho = \lambda \cdot \omega_{\pi(\rho)} \text{ for some } \lambda > 0 \}$ and $\Lambda^- := \{ \rho \in \lambda^m T^*M : \rho = \lambda \cdot \omega_{\pi(\rho)} \text{ for some } \lambda < 0 \}$. Notice $\Lambda^+ \cap \Lambda^- = \varnothing$. Then show $\Lambda^+$ and $\Lambda^-$ are path connected. Take $(p, \rho)$ and $(q, \tau)$. We may connect $\rho$ by a path to $\omega(p)$ where $\gamma(t) = (\lambda(1-t) + t)\omega(p) + \rho = \lambda\omega(p)$ where $\lambda > 0$. Then, $\omega(p)$ may be connected by a path with $\omega(q)$ by taking $\tilde{\gamma}(t)$ a path where $\tilde{\gamma}(0) = p$ and $\tilde{\gamma}(1) = q$ and put $\gamma(t) = \omega(\tilde{\gamma}(t))$. We do this so we can integrate on $M$.

\end{proof}

We have orientation so we may properly integrate. Reminder: $D \subset \mathbb{R}^n$ is open and bounded, $\phi: D \to \tilde{D} \subset \mathbb{R}^n$ is a diffeomorphism, and $f: \phi(D) \to \mathbb{R}$ a continuous function, then
\[
\int_{\phi(A)}f = \int_{A}f \circ \phi |\text{det}(D\phi)| \text{ (transformation formula)}
\]  
CHECK OUT PAGE 264 TU

Assume $\omega \in \Omega^n(\tilde{D})$. Then $\omega = f dx_1 \wedge \cdots \wedge dx_n$ for some $f \in C^\infty(\tilde{D})$. Put $\tilde{A} = \phi(A)$, and define
\[
\int_{\tilde(A)}\omega := \int_{\tilde{A}}f
\]
condsider $\phi^*(\omega) \in \Omega^n(D)$. Then $\phi^*(f dx_1 \wedge \cdots \wedge dx_n) = (f \circ \phi)\cdot \text{det}(D\phi) \cdot dx_1 \wedge \cdots \wedge dx_n$ 

Let $M$ be an oriented manifold and $\mathscr{A}$ an oriented atlas. Choose a partition of unity $(\phi_i)_{i \in \mathbb{N}}$ subordinate to $\mathscr{A}$. For each $\omega \in \Omega^m_c(M)$, put 
\[
\int_M \omega = \sum_{i \in \mathbb{N}} \int_{\tilde{U_i}} \phi_ix^{-1*}\omega
\]
Prove this is independent of atlas.

\section{March 11, 2015}
\textbf{Notations:}

(1) Denote $\mathbb{H}^n$ as the upper-half space which is the set $\{ (x_1, \ldots, x_n) \in \mathbb{R}^n | x_1 \geq 0 \}$.

(2) The interior of a manifold with boundary is denoted $M^\circ = M \sim \partial M$.

\textbf{Definition:} A \textbf{manifold with boundary} $M$ is a topolgical space which is Hausdorff and second countable subject to the following conditions: 

(1) A chart of $M$ in $\mathbb{H}^n$ is a homeomorphism $x: U \subset M \to \hat{U} \subset \mathbb{H}^n$ where $U, \hat{U}$ are open.

(2) Two charts $(U, x), (V,y)$ of $M$ in $\mathbb{H}^n$ are called $C^\infty$-compatible charts if $x \circ y^{-1}|_{U \cap V}: y(U \cap V) \to x(U \cap V)$ is a $C^\infty$-diffeomorphism.

(3) An atlas of $M$ in $\mathbb{H}^n$ consists of a set of $C^\infty$-compatible charts in $\mathbb{H}^n$ which cover $M$.

(4) A maximal atlas $\mathscr{A}$

\textbf{Definition:} Let $M$ be a manifold with boundary. Define $\partial M \subset M$ as the set of points $p \in M$ such that there is a chart $(U, x)$ around $p$ with $x(p) = (0, x_2(p), \ldots, x_n(p))$ with $p \in x^{-1}(\{ 0 \}\times \mathbb{R}^{n-1}$.

\textbf{Observations:}

(1) $\partial M$ is a manifold of dimension $n-1$. It's atlas is given by charts $(U\cap M, \bar{x}|_{U \cap \partial M})$ where $(U,x) \in \mathscr{A}$ and $p \in V$, with $\bar{x}(p) = (x_2(p), \ldots, x_n(p))$ (from $(0, x_2(p), \ldots, x_n(p))$). This gives us transition functions $\bar{x} \circ \bar{y}^{-1}: \bar{y}(U \cap V \cap \partial M) \to \bar{x}(U \cap V \cap \partial M)$ is a diffeomorphism.

(2) The tangent spaces of the interior are obvious. On the boundary, using curves ends up being very technical. The space of derivations definition gives a more obvious definition of the tangent space on the boundary. So, $T_pM = \text{Der}(C^\infty_p, \mathbb{R})$ for $p \in \partial M$. Then the tangent space is spanned by $\{ \frac{\partial}{\partial x_1}|_p, \ldots, \frac{\partial}{\partial x_n}|_p \}$.

(3) Orientation is defined in the same way. Notice that the boundary has an induced orientation. We get this from ...

\textbf{Theorem: (Stokes)} Given a compact oriented $m$-manifold $M$ with boundary. Then for each $\omega \in \Omega^{m-1}(M)$ Then
\[
\int_{\partial M} \omega|_{\partial M} = \int_M d\omega
\]
where $\partial M$ has the indeuced orientation.

\begin{proof} Recall the fundamental theorem of calculus:
\begin{align*}
\int_0^a \frac{\partial}{\partial s} f(s, t_2, \ldots, t_m) ds & = f(A, t_2, \ldots, t_m) - f(0, t_2, \ldots, t_m) \\
& = \int_{\{ A \}} f(t, t_2, \ldots, t_n) dt - \int_{\{ 0 \}}f(t, t_2, \ldots t_n)dt \\
\end{align*}
Now, let $Q \subset \mathbb{H}^n$ be a cube; that is, $Q = [a_1, b_1] \times \cdots [a_n, b_n]$ with $a_1 \geq 0$ and $a_2,\ldots a_n \in \mathbb{R}$, $b_i > a_i$ for all $i \in \{ 1, \ldots, n \}$. Let $\omega \in \Omega^{m-1}(Q)$ with the support of $\omega$ compactly contained in $Q$. Locally, we may represent $\omega$ as $\sum_{i=1}^m \omega_i dx_1 \wedge \cdots \wedge \hat{dx_i} \wedge \cdots \wedge dx_m$ for $\omega_i \in C^\infty(Q)$. Then
\begin{align*}
\int_Q d\omega & = \sum_i (-1)^i \int_Q \frac{\partial \omega_i}{\partial x_i}dx_1 \wedge \cdots \wedge dx_m \\
& = \sum_i \int_{Q_i}
\end{align*}

\end{proof}

\section{Friday, March 13}
We have $Q = [A_1, B_1] \times \cdots \times [A_n, B_n] \subset \mathbb{R}^n$, $\omega \in \Omega^{n-1}(M)$, and 
\[
\text{supp}(\omega) \subset \subset \begin{cases}
(A_1, B_1) \times \cdots \times (A_n, B_n) & A_1 > 0 \\
[0, B_1) \times (A_2, B_2) \times \cdots \times (A_n, B_n)
\end{cases}
\]
$\omega = \sum_i \omega_i dx_1 \wedge \cdots \wedge \hat{dx_i} \wedge \cdots \wedge dx_n$, $d\omega$ $= \sum_i (-1)^i\frac{\partial \omega_i}{\partial x_i} dx_1 \wedge \cdots dx_n$, and
\begin{align*}
\int_M d \omega & = \sum_i \int_{Q_i}(\int_{A_i}^{B_i} \frac{\partial \omega_i}{\partial x_i} dx_i )\wedge dx_1 \wedge \cdots \wedge \hat{dx_i} \wedge \cdots \wedge dx_n\\
& = -\int_{Q_1} \omega_1 dx_2 \wedge \cdots \wedge dx_n
\end{align*}
where $Q_i = [A_1, B_1] \times \cdots \times \hat{[A_i, B_i]} \times \cdots \times [A_n, B_n]$, because
\[
\int_{A_i}^{B_i} \frac{\partial \omega_i}{\partial x_i}dx_i = \omega_i(B_i) - \omega_i(A_i) = 0 - 0 = 0
\]
For the boundary, we have 
\[
\int_{\partial Q} \omega = \int_{Q_1} \omega = \int_{Q_1} -\omega_1 d\tilde{x}_2 \wedge \cdots \wedge d\tilde{x}_n
\]
We want an outward pointing orientation. Notice $\frac{\partial}{\partial x_1}$ points invward to $M$ (with respect to $Q$) but we want to orient $Q$, resp $M$ such that 
\[
-\frac{\partial}{\partial x_1}, \frac{\partial}{\partial x_2}, \ldots, \frac{\partial}{\partial x_n}
\]
Notice $\partial Q_1 \cup \cup_{l \geq 2} Q_l$

Now we choose an oriented atlas $\mathscr{A}$ of $M$, after passing to a finite atlas, we can assume that $\tilde{U} \subset \mathbb{R}^n$ for $(U,x) \in \mathscr{A}$ has form $[A_1, B_1] \times \cdots \times[A_n, B_n] \subset \mathbb{H}^n$ and $x: U \to \tilde{U} \subset \mathbb{H}^n$ and such that $\mathscr{A}$ is countable. Choose a smooth partition of unity. Choose a smooth partition of unity $(\phi_{(U,x)})_{(U,x) \in \mathscr{A}}$ subordinate to $\mathscr{A}$ where $\text{supp}(\phi_{(U,x)}) \subset \subset U \Rightarrow x_*(\phi_{(U,x)} \omega \subset \subset \tilde{U} \subset \mathbb{H}^n)$ and
\begin{align*}
\int_M d\omega &= \sum_{(U,x) \in \mathscr{A}} \int_{\tilde{U}}d(x_*\phi_{(U,x)} \omega) \\
&= \sum_{(U,x) \in \mathscr{A}} \int_{\partial \tilde{U}} x_*(\phi_{(U,x)}\omega)\\
&= \sum_{(U,x) \in \mathscr{A}} \int_{\partial \tilde{U}} x_*(\phi_{(U,x)}\omega|_{\partial M}) \\
&= \int_{\partial M} \omega
\end{align*}

\textbf{Corollary:} If $M$ is a closed manifold (compact and no boundary) then 
\[
\int_M d \omega = 0
\]
For $\omega \in \Omega^{\text{dim}M}(M)$

\textbf{Integration of Vector Fields}
Let $\xi: M \to TM$ be a $C^\infty$ vector field on a smooth manifold $M$. A curve $\gamma: I \to M$, $I = (a,b) \subset \mathbb{R}$ is called an integral curve of $M$ if $\xi(\gamma(t)) \in T_{\gamma(t)}M$ is equal to $\dot{\gamma}(t)$ for all $t \in I$.

\textbf{Question:}

\section{March 16, 2015}
\textbf{Integral Curves:} Let $\xi: M \to TM$ be a smooth vector field on a manifold $M$. By an integral curve of $\xi$, one understands a smooth map $\gamma: I \to M$, with $I \subset \mathbb{R}$ an open interval, such that
\[
\dot{\gamma}(t) = \xi(\gamma(t)) \text{ for all } t \in I
\]
\textbf{Observations:}

(1) For each $p \in M$, there exists an open interval $I \subset \mathbb{R}$ containing the origin 0, and a smooth integral curve $\gamma: I \to M$ of $\xi$ such that $\gamma(0) = p$.
\begin{proof}
Choose coordinates $(U,x)$ of $M$ around $p$, and then consider the following ordinary differential equation:
\begin{align*}
\dot{c}(t) = F(c(t)) & & c(0) = x(p)
\end{align*}
where $F := (pr_2 \circ Tc \circ \xi \circ x^{-1}): \hat{U} \to \mathbb{R}^n$. By existence and uniqueness (Picard-Lindelof theorem), there exists $c:(-\varepsilon, \varepsilon) \to \hat{U}$ such that the initial value problem is satisfied. We put $\gamma := x^{-1} \circ c: (-\varepsilon, \varepsilon) \to M$, then $\gamma(0) = p$ and $\dot{\gamma}(t) = (Tx)^{-1}(c(t), \dot{c}(t)) = Tx^{-1}(c(t), F(c(t))) = \xi(x^{-1}(c(t))) = \xi(\gamma(t))$. (Note that this holds true in Banach manifolds)
\end{proof}

(2) If $\gamma_1, \gamma_2$ are integral curves of $\xi$ with $\gamma_1(0) = \gamma_2(0) = p$, then $\gamma_1|_{I_1 \cap I_2} = \gamma_2|_{I_1 \cap I_2}$ (Note $I_1 \cap I_2$ is nonempty since they both implicitly contain 0)
\begin{proof}
Let $K = \{ t \in I_1 \cap I_2 : \gamma_1(t) = \gamma_2(t) \}$. We have $K = (\gamma_1, \gamma_2)^{-1}(\Delta_M)$. (note $(\gamma_1, \gamma_2): I_1 \cap I_2 \to M \times M$). By continuity of $\gamma_1$ and $\gamma_2$ and $M$ being Hausdorff, $I_1 \cap I_2$ is an open interval around the origin, hence connected. Let $t \in K$. Consider $\tilde{\gamma_1}: I_1 - t \to M$ and $\tilde{\gamma_2}: I_2 + t \to M$ where $\tilde{\gamma_i}(s) = \gamma_i(s+t)$. So $\tilde{\gamma_1}(0) = \gamma_1(t) = \gamma_2(t) = \tilde{\gamma_2}(0)$, so $\dot{\tilde{\gamma_i}}(s) = \dot{\gamma_i}(s+t) = \xi(\gamma_i(x + t)) = \xi(\tilde{\gamma_i}(s))$. By local uniqueness of the initial value problem, there exists an $\varepsilon$ such that $\tilde{\gamma_1}(s) = \tilde{\gamma_2}(s)$ for $s \in (-\varepsilon, \varepsilon)$. Hence $\gamma_1$ and $\gamma_2$ agree on an $\varepsilon$-neighborhood of $t$.
\end{proof}

(3) For each $p \in M$, let $I_p = (t_p^-, f_p^+)$ with $t_p^- < t_p^+$ and $t_p^-, t_p^+ \in \mathbb{R} \cup \{ \pm \infty \}$. There of all intervals $I$ such that there exists an integral curve $\gamma : I \to M$ of $\xi$ with $\gamma(0) = p$. Define $\gamma_p: I_p \to M$ by $t \mapsto \gamma(t)$, where $t \in I$ with $\gamma:I \to M$ (If $M$ is compact , the $I_p = \mathbb{R}$, a counterexample is the plane with a point removed and having a constant vector field oriented upwards). Now put $\mathcal{D} = \cup_{p \in M}I_p \times \{ p \} \subset \mathbb{R} \times M$, and $\phi:\mathcal{D} \to M$, $(t,p) \mapsto \gamma_p(t)$. Then $\phi$ is called the flow of the vector field $\xi$. It has the following nice properties:

(a) $\mathcal{D} \subset \mathbb{R} \times M$ is open.
 
(b) The domain $\phi_t \circ \phi_s \subset $ domain $\phi_{t+s}$ where $\phi_t: M \to M$ where $p \mapsto \phi(t,p)$

(c) $\phi_{t+ s}(p) = \phi_t \circ \phi_s(p)$ for $p \in \text{dom}(\phi_t \circ \phi_s)$.

(d) $\phi_d$
\begin{proof}
\end{proof}

\section{18 March, 2015 (Wednesday)}

\textbf{Banach Fixed Point Theorem:} If you have a complete metric space with a Lipschitz contraction, then the space has a unique fixed point.

\textbf{Proposition:} Let $J$ be an open interval containing $0$, $U$ an open set of a banach space $\mathbb{E}$, and $x_0 \in \mathbb{E}$. Let $a \in (0,1)$ such that the closed ball $\bar{B}_{3a} \subset U$. Assume that $f: J \times U \to \mathbb{E}$ be a bounded continuous map, bounded by constant $L \geq 1$, and satisfying on $U$ uniformly with respect to $J$ a Lipschitz condition with Lipschitz constant $K \geq 1$. Then $\| f(t,x) - f(t,y) \| \leq K \| x - y \|$ for all $t \in J$ and $x,y \in U$. If $b < \frac{a}{LK}$, then for each $x \in \bar{B}_a(x_0)$ there exists a unique flow $\phi: J_b \times B_a(x) \to U$; that is, $\frac{d}{dt}\phi(t,x) = f(t, \phi(t,x))$ and $\phi(0,x) = x$. Letting $I_b = [-b,b]$, and let $x$ be fixed in $\bar{B}_a(x_0)$. Let $M$ be a set of continuous maps
\[
a: I_b \to \bar{B}_{2a}(x_0)
\]
We have that $M$ is a complete metric space with distance given by the sup-norm.
\begin{align*}
S: M \to M && s\alpha(t) = x + \int_0^t f(u,\alpha(u))du
\end{align*}
Choose $S$ fulfills Lipschitz-condition with Lipschitz-constant $L_x < 1$ which implies there exists a unique fixed point by the Banach Fixed Point Theorem. Call this $\phi_x \in M$ with $s\phi_x = \phi_x$. By the fundamental theorem of calculus, we have $\phi_x(t) = x + \int_0^t f(u, \phi_x(u))du$ is differentiable; that is, $\dot{\phi_x}(t) = f(t,\phi_x(t))$ with $\phi_x(0) = x$. If $f$ is $C^k$ for $k \in \mathbb{N}^* \cup \{ +\infty \}$, then $\phi$ is $C^k$. Look at Lang Differentiable Manifolds for the full proof.

Last lecture we had $\phi: \mathcal{D} \to M$ by $(t,p) \mapsto \gamma_p(t)$. Then $\phi$ has the following properties:

(1) $\mathcal{D}\subset \mathbb{R} \times M$ is open

(2) $\text{dom}(\phi_s \circ \phi_t) \subset \text{dom}(\phi_{s + t})$ where $\phi_t :\mathcal{D} \cap \{t\} \times M = \mathcal{D}_y = \text{dom}(\phi_t)$ by $p \mapsto \phi(t,p)$

(3) We also have $\phi_{t+s} = \phi_t \circ \phi_s$ for $p \in \text{dom}(\phi_t \circ \phi_s)$

(4) $\phi_t: \mathcal{D}_t \to \mathcal{D}_{-t}$ is a diffeomorphism with inverse $\phi_{-t}$

\begin{proof}
(a) Local flow theorem from Lang

(b) Let $s \in (t_-(p),t_+(p))$ %and $t \in (t_-(\gamma_p(s)), t_+(\gamma_p(s)))$
Then $f \mapsto \gamma_p(s+t)$ is an integral curve of $\mathcal{G}$ and has maximal domain 
$(t_-(p) - s,t_+(p) - s) = (t_-(\gamma_p(s)), t_+(\gamma_p(s)))$.
Since $\gamma_p(s+0) = \gamma_p(s)$. Now 
$p \in \text{dom}(\phi_t \circ \phi_s) \Rightarrow p \in \text{dom}(\phi_s) \Rightarrow s \in (t_-(p), t_+(p))$
and $t \in (t_-(\gamma_p(s)),t_+(\gamma_p(s)) \Rightarrow t+s \in (t_-(p), t_+(p))$.

\end{proof}

\section{March 20, 2015 (Friday)}
\textbf{Lie Derivatives:} We want to take derivatives of vector fields $\xi: M \to TM$ which gives a tangent map $T\xi: TM \to T(TM)$. Assume $W: M \to TM$ is a second vector field. We want to define a derivative of $\xi$ with respect to $W$.

\textbf{Lie Derivative:} Looking at the flow of $W$, $\phi:\mathcal{D} \to M$ with
\[
\Lie_W \xi(p) := \lim_{t \to 0} \frac{T\phi_{-t}(\xi_{\phi_t(p)}) - \xi_p}{t} = \frac{d}{dt}T\phi_{-t}(\xi_{\phi_t(p)})|_{t=0}
\]
Notice that the limit exists in coordinates since all the functions are smooth. The map $\Lie_W$ is called the Lie derivative.\\
\textbf{Observations:}

(1) $\Lie_W f = W(f)$

(2) $\Lie_W \xi = [W, \xi]$

(3) $\Lie_W$ is tensorial in $W$ only over $\mathbb{R}$, not $C^{\infty}(M)$.

(4) $\Lie_W: \Omega^\bullet(M) \to \Omega^\bullet(M)$ commutes with $d$.

(5) $\Lie_W(\omega \wedge \rho) = \Lie_W\omega \wedge \rho + \omega \wedge \Lie_W \rho$

(6) \textbf{Cartan's Magical Formula:} $\Lie_W\omega = i_Wd\omega + di_W \omega$ for $\omega \in \Omega^k(M)$ where $i_W \in \Omega^{k-1}(M)$ is defined by $i_W\omega(Y_1, \ldots, Y_{k-1}) = \omega(W, Y_1, \ldots, Y_k)$ (useful for proving Poincare's lemma).
\begin{proof}
(1) $\Lie_Wf(p) = \frac{d}{dt}(\phi_t^*f)(p) = \frac{d}{dt}(f \circ \phi_t(p))|_{t = 0} = W(p)\cdot [f]_p$. \\
(2) We show that the bracket is a derivation to show that the bracket is still a vector field. \textbf{Exercise:} do this. \\
(3) Omitted \\
(4) We have $\Lie_W d \omega = \frac{d}{dt}\phi_t^*(d \omega) |_{t=0} = \frac{d}{dt}d(\phi_t^* \omega)|_{t=0} = d(\frac{d}{dt} \phi_t^*\omega)|_{t=0}$ \\
(5) Same argument as (4) \\
(6) We prove this by induction on $k$. For $k = 0$, $\Lie_Wf = Wf$ and $i_w df + di_Wf = i_wdf = Wf$. Assume this holds true for $k-1$.
\end{proof}

\section{March 30, 2015 (Monday)}
\textbf{Proposition:} $\mathcal{L}_XY = [X,Y]$
\begin{proof}
For $f \in C^\infty(M)$ we have
\begin{align*}
\mathcal{L}_XY(f) & = (\lim_{t \to 0} \frac{TX_{-t}Y_{x_y(m)} - y_m}{t})(f) \\
& = \frac{d}{dt}|_{t=0}(TX_{-t}Y_{X_t(m)})(t) \\
& = \frac{d}{dt}|_{t=0}Y_{X_t(m)}(f \circ X_{-t})
\end{align*}
For the auxillary function $H(t,u) = f(X_{-t}(Y_u(X_t(m))))$ with $(t,u) \in \mathbb{R}^2$, small enough. We have
\begin{align*}
Y_{X_t(m)}(f \circ X_{-t}) = \frac{\partial}{\partial r_2}|_{(t,0)}H(t,r_2) \\
%\mathcal{L}_Yg(p) = Y_g(p) = Y_pg 
\end{align*}
Then we have $\mathcal{L}_XY(f) = \frac{\partial^2}{\partial r_1 \partial r_2}|_{(0,0)}$. Consider another auxillary function $K(t,u,s) = f(X_s(Y_u(X_t(m))))$ we have $H(t,u) = K(t,u,-t)$
Then
\begin{align*}
\mathcal{L}_XY(f) & = \frac{\partial^2 K}{\partial r_1 \partial r_2}|_{(0,0,0)} - \frac{\partial^2 K}{\partial r_2 \partial r_3}|_{(0,0,0)} \\
\frac{\partial K}{\partial r_2}|_{(t,0,0)}& = Y_{X_t(m)}f = (Yf)(X_t(m)) \\
\frac{\partial^2 K}{\partial r_1 \partial r_2}|_{(0,0,0)} & = X_m(Yf) \\
\frac{\partial K}{\partial r_3}|_{(0,0,0)} & = Xf(Y_u(m)) \\
\frac{\partial^2 K}{\partial r_1 \partial r_3}|_{(0,0,0)} &= Y_m(Xf) 
\end{align*}
\end{proof}

\textbf{Cartan's Magic Formula} $\mathcal{L}_X \omega = i_X d \omega + d i_X \omega$ for $\omega \in \Omega^k(M)$.
\begin{proof}
This proof follows from induction. For $k = 0$
\[
\mathcal{L}_Xf = Xf = i_X df = i_X df + d i_X f
\]
Now, for the induction step, take
\begin{align*}
\mathcal{L}_X(df \wedge \omega) & = \Lie_Xdf \wedge \omega + df \wedge \Lie_X \omega \\
(i_Xd + di_X)(df \wedge \omega) & = -df \wedge i_Xd\omega + d(i_Xdf \wedge \omega - df \wedge i_X \omega) \\ 
& = - df \wedge i_X d \omega + d i_X df \wedge \omega + (i_Xdf)\wedge d \omega + df \wedge di_X\omega \\
& = \Lie_X df \wedge \omega  \cdots \text{ look in Tu }
\end{align*}

\textbf{Exercise:} Show $i_X(\rho \wedge \omega) = i_X \rho \wedge \omega + (-1)^{deg(\rho)} \rho \wedge i_X \omega$
\end{proof}

\section{April 1, 2015 (Wednesday)}

\textbf{Andy} Given a smooth n-manifold $M$, a \textbf{Riemannian metric} $g$ is a smooth symmetry covariant 2-tensor field on $M$ that is positive definite at each point in $M$; that is, $g \in \Gamma(T*M \otimes T*M)$. Locally, we may express $g$ as $g_{ij}dx^i \otimes dx^j$ for coordinates $(U, x^1, \ldots, x^n)$ where $(g_{ij})$ is a positive definite matrix of smooth functions.

A Kahler structure on a Riemannian manifold $(M^n, g)$ is given by a 2-form $\omega$ and a field of endomorphisms $J$ on the tangent bundle such that

Algebraic conditions:

(1) $J$ is an almost complex structure; that is, $J^2 = -Id$ as an endomorphism on the tangent space

(2) $g(X,Y) = g(JX, JY)$ for each $X,Y \in \Gamma(TM)$

(3) $\omega(X,Y)= g(JX,Y)$

Analytic conditions:

(4) The 2-form $\omega$ is closed; ie, $d\omega = 0$

(5) $J$ is integrable

Note that $(1)$ and $(5)$ are equivalent to having a holomorphic structure. If $N(X,Y) = 2([JX, JY] - [X,Y] - [JX, Y] - [X,JY]) = 0$ we have the holomorphic structure.

Locally, we may express $\omega$ as $ih_{\alpha \beta}dz_\alpha \wedge dz_{\bar{\beta}}$ where $h_{\alpha \beta} = h(\frac{\partial}{\partial z_{\alpha}}, \frac{\partial}{\partial z_{\bar{\beta}}})$ and $h$ is hermitian. Also, $\frac{\partial^2 u}{\partial z_\alpha \partial z_{\bar{\beta}}}$ where $u$ is the Kahler potential. As a side remark, the only solutions found to the Einstein vacuum equation $R_{\alpha \beta} = 0$ are Kahler manifolds.

A complex manifold is a smooth manifold of dimension $2n$ which admits a holomorphic atlas $\{U_i, \phi_i \}$ such that the transition functions $\phi_i$ are biholomorphic and map into $\mathbb{C}^n$. Remember that a functions $F = f + ig$ is holomorphic if it satisfies the Cauchy-Riemann equations
\begin{align*}
\frac{\partial f}{\partial x} = \frac{\partial g}{\partial y} && \frac{\partial f}{\partial y} = -\frac{\partial g}{\partial x}
\end{align*}

\textbf{Exercise:} Show that this is equivalent to the equation $\frac{\partial F}{\partial \bar{z}} = 0$

The canonical examples of a kahler manifolds are the complex projective plane, tori, $\mathbb{C}^n$, and Riemann surfaces. Note that every complex variety may be embedded in $\mathbb{CP}^n$.

\textbf{Nicholas:} A \textbf{Calabi-Yau manifold} is a compact Kahler manifold where the holonomy group is $SU(d)$ where $d$ is the complex dimension.

\textbf{Definition:} Take $C^\infty(M, TM)$ as the space of vector fields on $M$. A bilinear map $\nabla: C^\infty(M, TM) \to C^\infty(M, TM)$ where $(X,Y) \mapsto \nabla_X Y$ is a connection if it satisfies

(1) $\nabla_{fX}Y = f\nabla_XY$ for each $f \in C^\infty(M, TM)$

(2) $\nabla_X(fX) = X(f)Y + f\nabla_XY$

\textbf{Definition:} A vector field $X$ is parallel if $\nabla_YX = 0$ for every $Y \in C^\infty(M, TM)$

Take $\gamma: [a,b] \to M$ be a smooth curve on $M$. A vector field $X$ on $\gamma([a,b])$ is called a parallel transport of a vector $v \in T_{\gamma(a)M}$ if $\nabla_{\dot{\gamma(t)}}X = 0$ for each $t$ and $X(a) = v$.

If $X$ is a parallel transport of $v$ and $Y$ is a parallel transport of $w$ (both along $\gamma$) Then $c_1X + c_2Y$ is the unique parallel transport of $c_1v + c_2w$ along $\gamma$. Let $X^{e_i}$ be a parallel transport of $e_i$ along $\gamma$. Taking $f_\gamma:T_{\gamma(a)}M \to T_{\gamma(b)}M$ by $v = i^ie_i \mapsto i^i X^{e_i}$.

Considering all loops in $M$ based at $p \in M$. Taking $\alpha$ as a loop of $M$, the map $f_\alpha: T_{\gamma(a)} \to T_{\gamma(b)}M \in GL(n; \mathbb{R})$
\end{document}